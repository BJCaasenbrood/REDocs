\documentclass[12pt]{article}

\usepackage{arxiv}

\usepackage[utf8]{inputenc} % allow utf-8 input
\usepackage[T1]{fontenc}    % use 8-bit T1 fonts
\usepackage{hyperref}       % hyperlinks
\usepackage{url}            % simple URL typesetting
\usepackage{booktabs}       % professional-quality tables
\usepackage{amsfonts}       % blackboard math symbols
\usepackage{nicefrac}       % compact symbols for 1/2, etc.
\usepackage{microtype}      % microtypography
\usepackage{lipsum}
\usepackage{graphicx}
\graphicspath{ {./images/} }

\title{Proposal: On Research and Teaching \\ within Computational Dynamics (TU/e)}


\author{
dr.ir. Brandon Caasenbrood\\
  Department of Mechanical Engineering\\
  Dynamics and Control -- Computational Dynamics \\[0.5em]
  Eindhoven University of Technology (TU/e)\\
  Het Eeuwsel 6, 5612 AS Eindhoven, the Netherlands\\
  \url{b.j.caasenbrood@gmail.com}
  \vspace{-8mm}
}

\begin{document}
\maketitle
% \begin{abstract}

% \end{abstract}
\vspace{-10mm}
\section*{Research Proposal and Scientific Interest}
\hspace{5mm} \textbf{Co-Design: Morphology and Control:} Morphology, in the context of biology, is defined as the study of the form, structure, and material composition of organisms and their constituent parts, with emphasis on how these attributes determine function and facilitate environmental adaptation. This discipline investigates the relationship between anatomical features—such as shape, size, and material organization—and their functional and adaptive significance in evolutionary and ecological contexts. An excellent example of applied morphology in engineering is found in "\textit{soft robotics}". Soft robots, a modern subdomain of robotics, leverages on compliant materials to enhance adaptability and dexterity. The importance of morphology in facilitating control has been widely recognized within its research community, where designs are often inspired by biological systems that simplify controller design. 

\hspace{5mm} However, in the broader field of engineering, the deliberate integration of morphology into controller design processes remains rare, despite its critical role in the development of high-precision mechanical systems that should be driven by the increasing demand of the semiconductor industry. From a mechatronic perspective, the morphology of a mechanical system—encompassing its geometric configuration of degree of freedom, mass distribution, stiffness, and damping properties and thus constitutes the fundamental aspects of its dynamic behavior. These morphological parameters govern the system’s response to internal and external inputs and directly influence control design. Thus, while the terminology may differ, the underlying principle persists: \textbf{structure dictates function}, whether in the study of natural organism, soft robotics, or high precision systems. 

\hspace{5mm} Co-design represents a recent paradigm shift in system engineering framework, emphasizing the simultaneous optimization of a mechanical system’s morphology and its associated control to maximize performance. Unlike traditional design approaches—which often treat mechanical design and control as sequential, independent processes—co-design explicitly acknowledges their inherent interdependence. This integrated approach is grounded in the understanding that a system’s performance is not dictated solely by its physical properties (e.g., mass distribution, stiffness, or actuator placement) but also by how these properties interact with the controller’s dynamics. For example, many high-precision systems utilizing compliant mechanisms—structures that achieve motion and force transmission through elastic deformation—control inputs can directly modulate the system’s effective stiffness. Adjusting stiffness in this manner enables fine-tuning of system properties to suppress undesired vibrations or enhance positional accuracy. Conversely, ill-suited morphologies can impair control, e.g., insufficient disturbance rejection or energy inefficiency, regardless of algorithmic sophistication.

The primary objective of the research is to develop co-design frameworks that simultaneously integrate system morphology and control design, moving beyond traditional sequential optimization approaches or hand-driven design. The goal is to establish methodologies that holistically consider the interplay between physical structure and control strategies to achieve enhanced system performance.
While the principal application domains include (soft) robotics, high-precision (wafer) stage actuators, compliant mechanisms, dynamic linkages, the scope of co-design extends to a broad range of mechatronic and cyber-physical systems. Potential additional applications encompass biomedical devices (e.g., prosthetics and surgical robots), aerospace structures (e.g., morphing wings), and automotive systems. This plethora highlights the versatility and impact of integrated co-design methodologies across diverse engineering disciplines. 


% \hspace{5mm} \textbf{(Non-Smooth) Dynamics for Precision Systems:} Address stiffness disparities in high-tech systems (e.g., wafer stages with flexures) by combining your control background with the group’s non-smooth time integration schemes. Focus on mitigating vibrations/chattering induced by intermittent contact.

\hspace{5mm} \textbf{Computational Methods for Dynamic System Design:} 
Despite over the past three decades of research, despite exponential increases in computational processing power, surprisingly little tangible progress has been achieved beyond the very first co-designed robots (Sims, 1994). This stagnation is due in part to the nested complexity of evolving a systems’s morphology and learning a bespoke controller for every morphological variant.

The design of mechanical systems is inherently a dynamic process, involving the interplay of various physical phenomena such as forces, torques, and energy transfer. The complexity of these interactions necessitates the development of sophisticated computational methods to accurately model and simulate dynamic behavior. This is particularly crucial in high-precision applications, where even minor deviations can lead to significant performance degradation.
With the increasing complexity of mechanical systems, there is an increasing demand for the development of efficient computational methods for simulating dynamics. An example is the field of (humanoid) robotics, where extensive research and development are spent on simulation engines. Some prominent examples are Nvidia Isaac Sim, enabling researchers to simulate and test (AI-driven) robotics solutions in physics-based virtual environments. MuJoCo, a physics engine for robotics and biomechanics, which is widely used in reinforcement learning research -- which has been extended to Nvidia Wrap as to allow GPU acceleration. 

The fruits of these tools have been a 

This includes extending the group’s existing tools to handle large deformations and contact in soft robots, as well as integrating machine learning techniques to enhance simulation speed and accuracy.

Collaborate on real-time simulation of soft robots undergoing contact (e.g., elephant trunk wrapping around objects). Propose structure-preserving integration schemes (your PhD strength) for their nonlinear rod FEM, ensuring energy/momentum consistency during impact.

\section*{Education Plan}

\hspace{5mm} \textbf{Teaching philosophy:} Foster systems thinking by integrating design, modeling, and control in curricula. Highlight interdisciplinary connections (e.g., mechanics  control theory ↔ AI) through hands-on projects (e.g., designing a robot arm with co-optimized mechanics/control).

\textbf{Support curricula:} There are several courses that I have supported: Dynamics, Nonlinear Dynamics, Introduction to Robotics, Dynamics and Control of Mechanical Systems, and 


Courses/Labs to Develop:

    Co-Design of Mechanical Systems: Graduate course covering topology optimization, reduced-order modeling, and control co-design.

    Precision Mechatronics: Lab-based class on mitigating mechanical vibrations via control-aware design (e.g., flexure mechanisms with embedded actuators).

Outreach:
    Mentor students in cross-domain projects (e.g., soft robotics for medical applications) and collaborate with industry to address real-world co-design challenges (e.g., ASML, Siemens).

\bibliographystyle{unsrt}  
%\bibliography{references}  %%% Remove comment to use the external .bib file (using bibtex).
%%% and comment out the ``thebibliography'' section.


%%% Comment out this section when you \bibliography{references} is enabled.
\begin{thebibliography}{1}

\bibitem{Sims1994Jan} 
Karl Sims
\newblock Evolving 3D morphology and behavior by competition.
\newblock In {\em Artificial Life}, volume~1, number~4, pages 353--372. 
MIT Press, 1994.


\end{thebibliography}


\end{document}
