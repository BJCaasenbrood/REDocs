\textbf{Research abstract:} Traditional mechatronic system design has
largely followed a sequential approach, where mechanical structure and
control strategies are developed independently based on predefined
requirements. In this conventional paradigm, the mechanical design is
typically finalized before control considerations are introduced, often
neglecting the mutual influence between structural dynamics and control
performance. On the other hand, \emph{computational design}---or
\emph{co-design}---enables the concurrent optimization of both
mechanical structure and control algorithms. This integrated methodology
enhances overall system performance and adaptability, particularly in
high-precision applications where the interaction between structure and
control is of paramount importance. By leveraging on the methods of
co-design, the proposed research framework addresses the limitations of
traditional design approaches, facilitating improved motion accuracy,
disturbance rejection, requirement driven sensor-actuator placement, and
adaptability to varying operational conditions.

The research aims to develop a dynamic topology optimization framework
tailored for high-precision mechatronic systems. This approach extends
conventional density-based optimization by incorporating temporal
evolution, allowing for the simultaneous optimization of structural
configuration and control strategies. Utilizing multi-indexed density
variables across the spatiotemporal domain, the method efficiently
captures material distribution, actuator placement, and time-dependent
control inputs. Gradient-based optimization, combined with advanced
simulation techniques such as the material point method and automatic
differentiation, enables robust analysis and optimization of system
behavior. The anticipated outcomes include significant advancements in
the design and performance of next-generation mechatronic systems.
