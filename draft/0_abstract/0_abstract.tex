\textbf{Abstract:} \emph{Computional design}, also referred to as
\emph{co-design}, is recently emerging, transformative approach
optimizing structure and governing control law. Traditionally, the
design philosophy in mechatronic systems has been predominantly driven
by user requirements, with structural design and control treated as
sequential and largely independent processes. In this paradigm, the
mechanical design is established first (often without the consideration
control's impact on dynamics) and control strategies are subsequently
implemented to ensure robust system operation. Co-design, on the other
hand, enables the simultaneous refinement of both the physical
architecture and the control algorithms, thereby enhancing system
performance and robust environment adaptability. By leveraging dynamic
(topology) optimization, the proposed framework seeks to address the
limitations of conventional design methodologies, particularly in the
context of high-precision applications where the interplay between
structure and control are of paramount importance. The anticipated
outcomes include improved motion accuracy, vibration mitigation, and
adaptive responses to varying operational conditions, ultimately
advancing the capabilities of next-generation mechatronic systems.

The optimal configuration of a mechatronic system is inherently linked
to its operational dynamics, necessitating a co-design methodology that
simultaneously considers both morphology and control. To address these
challenges, the research proposal will explore dynamic topology
optimization framework tailored for high-precision mechatronic systems.
This approach extends conventional density-based optimization by
incorporating temporal evolution, enabling the concurrent optimization
of structural layout and actuation strategies for specific dynamic
tasks. By employing multi-indexed density variables across the
spatiotemporal domain, our method efficiently captures material
distribution, actuator placement, and time-dependent control inputs.
Gradient-based optimization techniques, combined with advanced
simulation methods such as the material point method and automatic
differentiation, facilitate robust forward and backward analysis of
system behavior. --\textgreater{}
