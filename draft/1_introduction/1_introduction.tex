\hypertarget{research-proposal}{%
\section{Research proposal}\label{research-proposal}}

\subsection*{Definitions}

Before detailing the topic of co-design, us first introduce some
definitions. \textbf{Morphology}, from the context of biology, refers to
the study of the form, structure, and material composition of organisms
and their constituent parts, with emphasis on how these attributes
determine function and facilitate environmental adaptation. This
discipline investigates the relationship between anatomical features ---
such as shape, size, and material organization --- and their functional
and adaptive significance in evolutionary and ecological contexts.
\textbf{Control}, on the other hand, encompasses algorithms by which the
system's dynamics achieve a desired behavior. It is often relies on
mathematical framework for designing control algorithms that govern
system responses, ensuring stability, accuracy, and robustness under
varying operational and environmental conditions.

Sensor placement and actuator placement are critical considerations.
Sensor placement involves determining the optimal locations for sensors
to maximize observability, minimize noise, and ensure accurate state
estimation. Strategic sensor positioning enables effective monitoring of
system variables, facilitating robust feedback and enhancing control
precision. Actuator placement, conversely, focuses on the spatial
arrangement of actuators to achieve desired force transmission,
responsiveness, and efficiency. The configuration of actuators directly
influences the system's controllability and dynamic performance,
affecting both stability and energy consumption. In advanced engineering
applications, the co-optimization of sensor and actuator placement is
increasingly recognized as essential for achieving superior system
performance, particularly in environments demanding high accuracy and
reliability.

An excellent example of applied morphology in engineering is found in
soft robotics. Soft robots, a modern subdomain of robotics, leverage
compliant materials to enhance adaptability and dexterity. The
importance of morphology in facilitating control has been widely
recognized within this research community, where designs are often
inspired by biological systems that simplify controller design. However,
in the broader field of engineering, the deliberate integration of
morphology into controller design processes remains rare, despite its
critical role in the development of \textbf{high-precision mechanical
systems} that are increasingly demanded by the semiconductor industry
and other advanced sectors. From a mechatronic perspective, the
morphology of a mechanical system---including:

\begin{itemize}
  \item Geometric configuration
  \item Degrees of freedom
  \item Mass distribution
  \item Stiffness and damping properties
\end{itemize}

\noindent that together constitutes the fundamental aspects of its
dynamic behavior. These morphological parameters govern the system's
response to internal and external inputs and directly influence control
design. Thus, while the terminology may differ across disciplines, the
underlying principle persists: \textbf{structure dictates function},
whether in the study of natural organisms, soft robotics, or
high-precision engineered systems.

\hypertarget{problem-statement}{%
\subsection{Problem statement}\label{problem-statement}}

To highlight the importance of co-design as a framework, let us consider
a few illustrative design problems.

\emph{Example 1 -- Disturbance-rejection}: Consider a mechanical
structured subject to external vibrations, such as those induced by
nearby machinery or environmental disturbances. The design objective is
to develop an active structure whose morphology and control strategy are
co-optimized to suppress the transmission of floor vibrations to the
payload.

\hypertarget{state-of-the-art-in-co-design}{%
\subsection{State-of-the-art in
co-design}\label{state-of-the-art-in-co-design}}

Co-design represents a recent paradigm shift in system engineering
frameworks, emphasizing the simultaneous optimization of a mechanical
system's morphology and its associated control strategies to maximize
performance. Unlike traditional design approaches---which often treat
mechanical design and control as sequential, independent
processes---co-design seeks to exploit the synergistic relationship
between structure and control, enabling the development of systems with
superior precision, adaptability, and efficiency.

\begin{figure*}[!h]
  \includegraphics[width=\textwidth]{sampleteaser.pdf}
  \caption{1907 Franklin Model D roadster. Photograph by Harris \&
    Ewing, Inc. [Public domain], via Wikimedia
    Commons. (\url{https://goo.gl/VLCRBB}).}
\end{figure*}

\lipsum[2-3]
